\documentclass[a4paper, article, oneside, 10pt]{memoir}

\usepackage{fapapersize}
\usefapapersize{
  210mm,297mm,
  35mm,35mm,
  30mm,30mm
}

\usepackage{indentfirst}

\usepackage{amsmath}
\usepackage{amssymb}

\usepackage{tikz}
\usetikzlibrary{%
  arrows,%
  arrows.meta,%
  calc,%
  shapes,%
  graphs,%
  positioning,%
}
\newcommand{\tikzinput}[1]{\input{tikz/#1.tikz}}
\tikzinput{syntaxdiagram}

\newcommand{\repruleinput}[1]{%
  \begin{minipage}{2in}%
    \centering%
    \vspace{2ex}%
    \tikzinput{reprule/#1}%
    \vspace{1ex}%
  \end{minipage}%
}

\usepackage{multirow}

\begin{document}
\pagestyle{ruled}
\chapter{Context-free Grammar for CUTE Lang}
This chapter explains what context-free grammar is used in CUTE Lang.

\section{Syntax Diagram}
To explain efficiently, grammar is explained by syntax diagram first.

\subsection{Syntax Diagram Representation Rules}
Representation rules for our syntax diagram are on Table~\ref{tab:Representation_rules_of_syntax_diagram}.

\begin{table}[!htb]
  \centering
  \begin{tabular}{ccc}
    \cline{1-3}
    Repesentation & EBNF Equivalent & Meaning\\
    \cline{1-3}
    \repruleinput{nonterminal} & x & Nonterminal\\ \cline{1-3}
    \repruleinput{nonterminalex} & x \--- y & Nonterminal except $y$\\ \cline{1-3}
    \repruleinput{terminal} & ``x'' & Terminal\\ \cline{1-3}
    \repruleinput{sequence} & a, b & Sequence\\ \cline{1-3}
    \repruleinput{selection} & a \textbar{} b & Selections\\ \cline{1-3}
    \repruleinput{optional} & [ x ] & Optional\\ \cline{1-3}
    \repruleinput{zerorepeat} & \{ x \} & Zero or more repeat\\ \cline{1-3}
    \repruleinput{onerepeat} & x, \{ x \} & One or more repeat\\ \cline{1-3}
  \end{tabular}
  \caption{Representation rules of syntax diagram}
  \label{tab:Representation_rules_of_syntax_diagram}
\end{table}

\subsection{Syntax Diagram for Context-free Grammar}
Context-free grammar expressed by syntax diagram is the following.

%%%%%%%%%%%%%%%%%%%%%%%%%%%%%%%%%%%%%%%%%%%%%%%%%%%%%%%%%%%%
% If we use `top def' instead of `top st',
% then we can't seperate `module def' as `module st'.
% Does we need to set the level of `module def'
% to same level of `top f def'?
%%%%%%%%%%%%%%%%%%%%%%%%%%%%%%%%%%%%%%%%%%%%%%%%%%%%%%%%%%%%
\noindent
\begin{tikzpicture}[syntaxdiagram]
  \graph{
    top st[definition],
    i0 -!- i1 -!-
    start --[start] p0 --
    p1 --[vh way=0.5mm]
    {
      top f def p0 -> top f def[nonterminal] --
      top f def p1,
      %
      top f typing p0 -> top f typing[nonterminal] --
      top f typing p1,
    } --[hv way=0.5mm]
    p2 -> p3 --[end] end;

    p1 -- top f def p0;
    top f def p1 -- p2;
  };
\end{tikzpicture}

\noindent
\begin{tikzpicture}[syntaxdiagram]
  \graph{
    top f def0/top f def[definition],
    i0 -!- i1 -!-
    start --[start] p0 --
    p1 -> f def[nonterminal] --
    p2 -- p3 -> with st[nonterminal] --
    p4 --
    p5 -> p6 --[end] end;

    p2 --[skip way=0.5mm/-7mm] p5;
  };
\end{tikzpicture}

\noindent
\begin{tikzpicture}[syntaxdiagram]
  \graph{
    f defs0/f defs[definition],
    i0 -!- i1 -!-
    start --[start] p0 --
    p1 -> curlybo0/\{[terminal] --
    p2 --[vh way=0.5mm]
    {
      f def[nonterminal],
      comma0/";"[terminal],
    } --[hv way=0.5mm]
    p3 --
    p4 -- p5 -> comma1/";"[terminal] --
    p6 --
    p7 -> curlybc0/\}[terminal] --
    p8 -> p9 --[end] end;

    p2 -> f def -- p3;
    p3 ->[vh way=0.5mm]
    comma0 --[hv way=0.5mm] p2;

    p4 --[skip way=0.5mm/-7mm] p7;
  };
\end{tikzpicture}

\noindent
\begin{tikzpicture}[syntaxdiagram]
  \graph{
    f def0/f def[definition],
    i0 -!- i1 -!-
    start --[start] p0 --
    p1 -> f head[nonterminal] --
    p2 -> f body[nonterminal] --
    p3 -> p4 --[end] end;
  };
\end{tikzpicture}

\noindent
\begin{tikzpicture}[syntaxdiagram]
  \graph{
    f head0/f head[definition],
    i0 -!- i1 -!-
    start --[start] p0 -> var[nonterminal] --
    p1 --[vh way=0.5mm]
    {
      param space0,
      param id0/id[nonterminal],
    } --[hv way=0.5mm]
    p2 -> coloneq0/":="[terminal] --
    p3 -> p4 --[end] end;

    p1 -- param space0 -- p2;
    p2 ->[vh way=0.5mm]
    param id0 --[hv way=0.5mm] p1;
  };
\end{tikzpicture}

\noindent
\begin{tikzpicture}[syntaxdiagram]
  \graph{
    f body0/f body[definition],
    i0 -!- i1 -!-
    start --[start] p0 --
    p1 -> expr[nonterminal] --
    p2 -> p3 --[end] end;
  };
\end{tikzpicture}

\noindent
\begin{tikzpicture}[syntaxdiagram]
  \graph{
    with st0/with st[definition],
    i0 -!- i1 -!-
    start --[start] p0 --
    p1 -> with[terminal] --
    p2 -> f defs[nonterminal] --
    p3 -> p4 --[end] end;
  };
\end{tikzpicture}

\noindent
\begin{tikzpicture}[syntaxdiagram]
  \graph{
    expr0/expr[definition],
    i0 -!- i1 -!-
    start --[start] p0 --
    p1 -> infix expr[nonterminal] --
    p2 -- p3 -> ":"[terminal] --
    p4 -> type[nonterminal] --
    p5 --
    p6 -> p7 --[end] end;

    p2 --[skip way=0.5mm/-7mm] p6;
  };
\end{tikzpicture}

%%%%%%%%%%%%%%%%%%%%%%%%%%%%%%%%%%%%%%%%%%%%%%%%%%%%%%%%%%%%
%  If [If expr] is an left expression,
% we cannot deterministically parse following statement.
% ```
%  if true then 1 else 2 + 3
% ```
%  This is same for [Lambda expr], [Let expr],
% and etc. too.
%%%%%%%%%%%%%%%%%%%%%%%%%%%%%%%%%%%%%%%%%%%%%%%%%%%%%%%%%%%%

\noindent
\begin{tikzpicture}[syntaxdiagram]
  \graph{
    infixexpr0/infix expr[definition],
    i0 -!- i1 -!-
    start --[start] p0 --
    p1 --[vh way=0.5mm]
    {
      bi op p0 -> bi op left expr0/left expr[nonterminal] --
      bi op p1 -> bi op bi op0/bi op[nonterminal] --
      bi op p2 -> bi op infix expr0/infix expr[nonterminal] --
      bi op p3,
      %
      un op p0 -> un op un op0/un op[nonterminal] --
      un op p1 -> un op left expr0/left expr[nonterminal] --
      un op p2,
      %
      left expr p0 -> left expr left expr0/left expr[nonterminal] --
      left expr p1,
      %
      if p0 -> if[terminal] --
      if p1 -> if expr0/expr[nonterminal] --
      if p2 -> then[terminal] --
      if p3 -> if expr1/expr[nonterminal] --
      if p4 -> else[terminal] --
      if p5 -> if expr2/expr[nonterminal] --
      if p6,
      % 
      let p0 -> let[terminal] --
      let p1 -> let f defs0/f defs[nonterminal] --
      let p2 -> in[terminal] --
      let p3 -> let expr0/expr[nonterminal] --
      let p4,
      %
      lam p0 -> "$\lambda$"[terminal] --
      lam p1 -> lam id0/id[nonterminal] --
      lam p2 -> "$\rightarrow$"[terminal] --
      lam p3 -> lam expr0/expr[nonterminal] --
      lam p4,
      lam space0,
    } --[hv way=0.5mm]
    p2 -> p3 --[end] end;

    p1 -- bi op p0;
    bi op p3 -- p2;

    p1 -!- lam space0 -!- p2;
    lam p2 --[rep way=0.5mm/-7mm] lam p1;
  };
\end{tikzpicture}

\noindent
\begin{tikzpicture}[syntaxdiagram]
  \graph{
    left expr0/left expr[definition],
    i0 -!- i1 -!-
    start --[start] p0 --
    p1 -> f expr[nonterminal] --
    p2 -> p3 --[end] end;
  };
\end{tikzpicture}

\noindent
\begin{tikzpicture}[syntaxdiagram]
  \graph{
    f expr0/f expr[definition],
    i0 -!- i1 -!-
    start --[start] p0 --
    p1 -- p2 -> f expr1/f expr[nonterminal] -- p3 --
    p4 -> a expr[nonterminal] --
    p5 -> p6 --[end] end;

    p1 --[skip way=0.5mm/-7mm] p4;
  };
\end{tikzpicture}

\noindent
\begin{tikzpicture}[syntaxdiagram]
  \graph{
    a expr0/a expr[definition],
    i0 -!- i1 -!-
    start --[start] p0 --
    p1 --[vh way=0.5mm]
    {
      var p0 -> var var0/var[nonterminal] --
      var p1,
      %
      value literal p0 -> value literal value literal0/value literal[nonterminal] --
      value literal p1,
      %
      list p0 -> list squarebo0/"["[terminal] --
      list p1 --[vh way=0.5mm]
      {
        list normal p0 --[vh way=0.5mm]
        {
          list normal expr0/expr[nonterminal],
          list normal comma0/","[terminal],
        } --[hv way=0.5mm]
        list normal p1 --
        list normal p2 -- list normal p3 -> list normal comma1/","[terminal] --
        list normal p4 --
        list normal p5 -> list normal p6,
        %
        list enum p0 -> list enum expr0/expr[nonterminal] --
        list enum p1 -- list enum p2 -> list enum comma0/","[terminal] --
        list enum p3 -> list enum expr1/expr[nonterminal] --
        list enum p4 --
        list enum p5 -> list enum dotdot0/".."[terminal] --
        list enum p6 -> list enum expr2/expr[nonterminal] --
        list enum p7,
        list enum space0,
      } --[hv way=0.5mm]
      list p2 -> list squarebc0/"]"[terminal] --
      list p3,
      %
      section p0 -> section roundbo0/"("[terminal] --
      section p1 --[vh way=0.5mm]
      {
        no section p0 -> no section expr0/expr[nonterminal] --
        no section p1,
        %
        left section p0 -> left section infix expr0/infix expr[nonterminal] --
        left section p1 -> left section bi op0/bi op[nonterminal] --
        left section p2,
        %
        right section p0 -> right section bi op0/bi op${}_{/un op}$[nonterminal] --
        right section p1 -> right section infix expr0/infix expr[nonterminal] --
        right section p2,
      } --[hv way=0.5mm]
      section p2 -> section roundbc0/")"[terminal] --
      section p3,
    } --[hv way=0.5mm]
    p2 -> p3 --[end] end;

    p1 -- var p0;
    var p1 -- p2;

    p1 -!- list space0 -!- p2;
    list p1 -- list normal p0;
    list normal p6 -- list p2;

    list normal p0 -> list normal expr0 -- list normal p1;
    list normal p1 ->[vh way=0.5mm]
    list normal comma0 --[hv way=0.5mm] list normal p0;
    list normal p2 --[skip way=0.5mm/-7mm] list normal p5;

    list p1 -!- list enum space0 -!- list p2;
    list enum p1 --[skip way=0.5mm/-7mm] list enum p5;

    section p1 -- no section p0;
    no section p1 -- section p2;
  };
\end{tikzpicture}

\noindent
\begin{tikzpicture}[syntaxdiagram]
  \graph{
    top f typing0/top f typing[definition],
    i0 -!- i1 -!-
    start --[start] p0 --
    p1 -- p2 -> var[nonterminal] --
    p3 -> colon0/":"[terminal] --
    p4 -> type0/type[nonterminal] --
    p5 -> p6 --[end] end;
  };
\end{tikzpicture}

\noindent
\begin{tikzpicture}[syntaxdiagram]
  \graph{
    type0/type[definition],
    i0 -!- i1 -!-
    start --[start] p0 --
    p1 -- p2 -> "$\forall$"[terminal] --
    p3 -> var[nonterminal] --
    p4 -> ","[terminal] --
    p5 --
    p6 -> f type[nonterminal] --
    p7 -> p8 --[end] end;

    p1 --[skip way=0.5mm/-10.5mm] p6;
    p4 --[rep way=0.5mm/-7mm] p3;
  };
\end{tikzpicture}

\noindent
\begin{tikzpicture}[syntaxdiagram]
  \graph{
    f type0/f type[definition],
    i0 -!- i1 -!-
    start --[start] p0 --
    p1 -> c type[nonterminal] --
    p2 -- p3 -> "$\rightarrow$"[terminal] --
    p4 -> f type1/f type[nonterminal] --
    p5 --
    p6 -> p7 --[end] end;

    p2 --[skip way=0.5mm/-7mm] p6
  };
\end{tikzpicture}

\noindent
\begin{tikzpicture}[syntaxdiagram]
  \graph{
    c type0/c type[definition],
    i0 -!- i1 -!-
    start --[start] p0 --
    p1 -- p2 -> c type1/c type[nonterminal] -- p3 --
    p4 -> a type[nonterminal] --
    p5 -> p6 --[end] end;

    p1 --[skip way=0.5mm/-7mm] p4;
  };
\end{tikzpicture}

\noindent
\begin{tikzpicture}[syntaxdiagram]
  \graph{
    a type0/a type[definition],
    i0 -!- i1 -!-
    start --[start] p0 --
    p1 --[vh way=0.5mm]
    {
      type var p0 -> type var var0/var[nonterminal] --
      type var p1,
      %
      type literal p0 -> type literal type literal0/type literal[nonterminal] --
      type literal p1,
      %
      type prec p0 -> type prec roundbo0/"("[terminal] --
      type prec p1 -> type prec type0/type[nonterminal] --
      type prec p2 -> type prec roundbc0/")"[terminal] --
      type prec p3,
    } --[hv way=0.5mm]
    p2 -> p3 --[end] end;

    p1 -- type var p0;
    type var p1 -- p2;
  };
\end{tikzpicture}

\noindent
\begin{tikzpicture}[syntaxdiagram]
  \graph{
    var0/var[definition],
    i0 -!- i1 -!-
    start --[start] p0 --
    p1 --[vh way=0.5mm]
    {
      var id p0 -> var id0/id[nonterminal] --
      var id p1,
      var op p0 -> var op roundo0/"("[terminal] --
      var op p1 -> var op op0/op[nonterminal] --
      var op p2 -> var op roundc0/")"[terminal] --
      var op p3,
    } --[hv way=0.5mm]
    p2 -> p3 --[end] end;

    p1 -- var id p0;
    var id p1 -- p2;
  };
\end{tikzpicture}

\noindent
\begin{tikzpicture}[syntaxdiagram]
  \graph{
    op0/op[definition],
    i0 -!- i1 -!-
    start --[start] p0 --
    p1 --[vh way=0.5mm]
    {
      bi op p0 -> bi op0/bi op[nonterminal] --
      bi op p1,
      un op p0 -> un op0/un op[nonterminal] --
      un op p1,
    } --[hv way=0.5mm]
    p2 -> p3 --[end] end;

    p1 -- bi op p0;
    bi op p1 -- p2;
  };
\end{tikzpicture}

\noindent
\begin{tikzpicture}[syntaxdiagram]
  \graph{
    bi op0/bi op[definition],
    i0 -!- i1 -!-
    start --[start] p0 --
    p1 --[vh way=0.5mm]
    {
      bi sym p0 -> bi sym0/bi sym[nonterminal] --
      bi sym p1,
      bi fun p0 -> bi fun backquote0/"${}^\backprime$"[terminal] --
      bi fun p1 -> bi fun var0/var[nonterminal] --
      bi fun p2 -> bi fun backquote1/"${}^\backprime$"[terminal] --
      bi fun p3
    } --[hv way=0.5mm]
    p2 -> p3 --[end] end;

    p1 -- bi sym p0;
    bi sym p1 -- p2;
  };
\end{tikzpicture}

\noindent
\begin{tikzpicture}[syntaxdiagram]
  \graph{
    un op0/un op[definition],
    i0 -!- i1 -!-
    start --[start] p0 --
    p1 -> un sym[nonterminal] --
    p2 -> p3 --[end] end;
  };
\end{tikzpicture}

\end{document}
%
%%% Local Variables:
%%% TeX-engine: xetex
%%% End: